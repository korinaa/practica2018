\documentclass{report}
\usepackage[utf8]{inputenc}
\usepackage{ucs}
\usepackage{color}
\usepackage{graphicx}
\usepackage{subfig}
\usepackage[english,romanian]{babel}
\begin{document}
\begin{titlepage}
 \begin{figure}
\includegraphics[width=0.2\textwidth]{sigla}\\
\end{figure}
\begin{center}
    \textsc{\normalsize Universitatea "Ovidius" Constanța\\
  Facultatea de Matematică și Informatică\\
  Specializarea Informatică}\\
  [4cm]
    {\Large \sc Raport practică: 25.06-06.07.2018}\\[7cm]
    \begin{flushright} \large
    \textbf{Student:\ \ \ \ \ } \\
    Tudorică Anca-Maria
   \end{flushright}
  \vfill
  {\large Constanța}\\
  {\large Iunie-Iulie 2018}
 \end{center}
\end{titlepage}

\tableofcontents
\chapter{Introducere}

Acest raport cuprinde descrierea activității desfășurate la practică la calculator în perioada 25.06-06.07.2018 în cadrul Facultății de Matematică și Informatică.

\chapter{Activități planificate}
\begin{itemize}
\item  Luni, 25.06.2018 \newline
Aducerea la cunoștință a obiectivelor și cerințelor practicii la calculator.
\item  Marți, 26.06.2018 \newline
Am ales tema: "Algoritmul de căutare în adâncime într-un graf".
\item  Miercuri, 27.06.2018 \newline
Am studiat și am practicat Latex: am realizat prima pagina a lucrarii și am stabilit activitățile pe care doresc să le parcurg.
\item  Joi, 28.06.2018 \newline
Am căutat informații referitoare la grafuri și parcurgerea grafurilor.
\item  Vineri, 29.06.2018  \newline
Am rezolvat două exemple pentru căutarea în adâncime  într-un graf.
\item  Luni, 02.07.2018  \newline
Am căutat pseudocodul pentru tema aleasa.
\item  Marți, 03.07.2018  \newline
Am scris codul în limbajul java.
\item  Miercuri, 04.07.2018  \newline
Am verificat funcționalitatea codului. 
\item  Joi, 05.07.2018  \newline
Am completat raportul în latex cu etapele pe care le-am parcurs în zilele precedente.
\item  Vineri, 06.07.2018  \newline
Prezentarea lucrărilor.
Notarea finală a activității.
\end{itemize}

\chapter{25.06.2018}
Am desfăţurat următoarele activităţi:
\begin{itemize}
\item
Am identificat sursele pentru MikTeX, Git, SmartGit și BitBucket.
\begin{itemize}
\item
Am identificat sursele pentru MikTeX, Git, SmartGit și BitBucket.
\item
Am instalat, configurat pe calculatorul de lucru aplicațiile necesare:
\begin{itemize}
\item
MikTeX
\item
SmartGit
\item
Bitbucket
\end{itemize}
\item
Am instalat, configurat pe calculatorul de lucru aplicațiile necesare:
\begin{itemize}
\item
MikTeX
\item
SmartGit
\item
Bitbucket
\end{itemize}
\end{itemize}
\end{itemize}
Studierea obiectivelor și cerințelor față de practica de producție. Clarificarea situațiilor incerte.


\chapter{26.06.2018}
Am ales tema: "Algoritmul de cautare în adancime într-un graf".
\chapter{27.06.2018}
Am studiat și am practicat Latex: am realizat prima pagina a lucrarii și am stabilit activitătile pe care doresc să le parcurg.


\end{document}
